\documentclass[a4paper]{article}
\usepackage[utf8]{inputenc}
\inputencoding{utf8}
\usepackage[bulgarian]{babel}
\selectlanguage{bulgarian}

\usepackage{amsmath}
\usepackage[margin=1in]{geometry}
\title{Домашно 1 по \\* ``Ламбда смятане и теория на доказателствата''}
\author{Цветан Цветанов, студент 3 курс, ИС, ФМИ}

\begin{document}

\maketitle
\thispagestyle{empty}
\newpage

\section*{Задача 1}

Докажете, че двете дефиниции за числа на Хеминг са еквивалентни:

\begin{align*}
H_{1} = \{ x \mid p/x, p \in \{2, 3, 5\}\} \iff \quad H_{2}: \quad & 1)\; 1 \in H_{2} \\
                                                                   & 2)\; h \in H_{2} \implies 2h, 3h, 5h \in H_{2}
\end{align*}

Д-во:
\begin{align*}
& 1)\; 1 \in H_{2} \implies 1 \in H_{1}, \text{ защото 1 дели 2, 3 и 5 } \\
& 2)\; h \in H_{2} \implies \\ 
      & \qquad 2h \in H_{2}, \text{ но } 2/2h = 1/h \implies h \in H_{1} \text{ или} \\
      & \qquad 3h \in H_{2}, \text{ но } 3/3h = 1/h \implies h \in H_{1} \text{ или} \\
      & \qquad 5h \in H_{2}, \text{ но } 5/5h = 1/h \implies h \in H_{1} \\
      & \implies H_{2} \subseteq H_{1} \\
& \text{Аналогично } H_{1} \subseteq H_{2} \implies H_{1} = H_{2} \\
Q.E.D.
\end{align*}

\section*{Задача 2}

Да се докаже, че с индукция по дефиниция се построява графиката на тотална функция \\
$f : X \rightarrow Y, F$

Д-во:
\begin{align*}
& dom F = \{ x | f(x) \in Y, x \in X \} \implies F \subseteq X \implies F \in G_f
\end{align*}

\end{document}
