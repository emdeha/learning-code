\documentclass[a4paper]{article}
\usepackage[utf8]{inputenc}
\inputencoding{utf8}
\usepackage[bulgarian]{babel}
\selectlanguage{bulgarian}

\usepackage{titling}
\newcommand{\subtitle}[1]{%
  \posttitle{%
    \par\end{center}
    \begin{center}\large#1\end{center}
    \vskip0.5em}%
}

\usepackage{amsmath}
\usepackage[margin=1in]{geometry}
\title{Документация към проект по \\* ``Системи основани на знания''}
\subtitle{``Препоърчване на кинга''}
\author{Цветан Цветанов, студент 3 курс, ИС, ФМИ}

\begin{document}

\maketitle
\thispagestyle{empty}
\newpage

\section{Описание.}

Следната система основана на знания има за цел да препоръчва книги на 
определени типове хора.  Хората могат да са в три категории - бизнесмени,
студенти или бедняци.  Всеки човек се определя от някакви полезни за него
характеристики, на базата на които се прави заключение за това каква книга 
би му била полезна.  Нормално е за някой тип хора, например студентите, 
които не учат и ходят твърде често на партита, системата да не може да
препоръча книга.  В крайна сметка, това не би направило щастливи този тип
хора.

Системата работи като първо определя какъв тип е човека:
\begin{itemize}
\item determine-type - определя типа на човека; оттук стартираме
\item determine-business - доспецифицира бизнес човек
\item determine-student - доспецифицира студент
\item determine-poor - доспецифицира бедняк
\end{itemize}

След това, на базата на разни параметри от типа човек, системата определя
каква книга да му препоръча:
\begin{itemize}
\item suggest-book-poor - препоръчва книга на бедняк
\item suggest-book-working - препоръчва книга на работещ студент
\item suggest-book-studying - препоръчва книга на учещ студент
\item suggest-book-not-studying - препоърчва книга на неучещ студент
\item suggest-book-startup-founder - препоръчва книга на основател на стартъп
\item suggest-book-businessman - препоръчва книга на бизнесмен
\end{itemize}

\section{Определяне на прости факти.}
\begin{verbatim}правило determine-type\end{verbatim}

Опитваме се да определим прости факти за човека, а също така и какъв е по
професия - Бедняк, Бизнесмен или Студент.
\section{Общи функции.}
\begin{verbatim}функция determine-business\end{verbatim}

Определя какъв бизнес има даден човек.
\section{Правила за бизнесмен.}
\begin{verbatim}правило determine-businessman\end{verbatim}

Това правило дохарактеризира бизнесмен.
\begin{verbatim}функция businessman-query\end{verbatim}

Тази функция определя евристиката за препоръчване на книга на бизнесмен.
\begin{verbatim}правило suggest-book-businessman\end{verbatim}

Това правило препоръчва книга на бизнесмен на базата на евристиката
`businessman-query`.
\section{Правила и функции за студент}
\begin{verbatim}функция determine-working\end{verbatim}

Тази функция има за цел да характеризира работещ студент.
\begin{verbatim}функция determine-ordinary\end{verbatim}

Това е помощна функция за суперкласа `Ordinary` студент.

672]" crlf)
\begin{verbatim}функция determine-studying\end{verbatim}

Тази функция определя учещ студент с помощта на `determine-ordinary`.

24]" crlf)
\begin{verbatim}функция determine-not-studying\end{verbatim}

Тази функция определя неучещ студент с помощта на `determine-ordinary`.

100]" crlf)
\begin{verbatim}функция determine-startup-founder\end{verbatim}

Тази функция определя основател на стартъп.

В тази глобална променлива се държат възможните типове студенти с цел
предоставянето на списък за избор на човека, на който се препоръчва книга.
\begin{verbatim}правило determine-student\end{verbatim}

Това правило дохарактеризира студент.
\begin{verbatim}функция working-query\end{verbatim}

Тази функция определя евристиката за препоръчване на книга 
на работещ студент.
\begin{verbatim}правило suggest-book-working-student\end{verbatim}

Това правило препоръчва книга на работещ студент на базата на евристиката
`working-query`.
\begin{verbatim}функция studying-query\end{verbatim}

Тази функция определя евристиката за препоръчване на книга на
учещ студент.
\begin{verbatim}правило suggest-book-studying-student\end{verbatim}

Това правило препоръчва книга на учещ студент на базата на евристиката
`studying-query`.
\begin{verbatim}функция not-studying-query\end{verbatim}

Тази функция определя евристиката за препоръчване на книга на
неучещ студент.
\begin{verbatim}правило suggest-book-not-studying-student\end{verbatim}

Това правило препоръчва книга на неучещ студент на базата на евристиката
`not-studying-query`.
\begin{verbatim}функция startup-founder-query\end{verbatim}

Тази функция определя евристиката за препоръчване на книга на
основател на стартъп.
\begin{verbatim}правило suggest-book-startup-founder\end{verbatim}

Това правило препоръчва книга на основател на стартъп на базата на 
евристиката `startup-founder-query`.
\section{Правила за бедняк.}
\begin{verbatim}правило determine-poor\end{verbatim}

Това правило характеризира бедняк.
\begin{verbatim}функция poor-query\end{verbatim}

Тази функция определя евристиката за препоръчване на книга на бедняк.
\begin{verbatim}правило suggest-book-poor\end{verbatim}

Това правило препоръчва книга на бедняк на базата на евристиката 
`poor-query`.

\end{document}
